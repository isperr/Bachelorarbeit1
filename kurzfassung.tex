Diese Bachelorarbeit bietet einen Einblick in das Beobachtungsverhalten eines Webbenutzers, wenn dieser im Internet eine Suchmaschine benutzt. Um besser zu verstehen, wie verschiedene User eine Suchergebnisseite (search engine result page -- SERP) einer üblichen Suchmaschine betrachten und verwenden, kann ein Eye Tracking Gerät angebracht werden. Dieses zeichnet die Augenbewegungen eines Benutzers auf, damit die Daten zu einem späteren Zeitpunkt ausgewertet werden können \autocite{liu2015influence}. Daher ist es notwendig, die relevantesten Eye Tracking Technologien, Parameter und Visualisierungsmethoden näher zu betrachten. Darüber hinaus konzentriert sich diese Arbeit auf Suchergebnisseiten, wie sie aufgebaut sind und wie der Benutzer die dort präsentierten Informationen erfasst und verarbeitet.
Um besser zu verstehen wie SERPs aufgebaut sind, wie die Ergebnisseite näher an einem Beispiel einer modernen kommerziellen Suchmaschine erklärt.
Darüber hinaus gibt es einen Vergleich moderner Suchmaschinen, der die Vorteile und Nachteile der SERP-Struktur betont.
Nur wenn diese Anforderungen erfüllt werden, kann eine Suchergebnisseite verbessert und entsprechend angepasst werden, um die Effizienz der Informationsgewinnung zu maximieren.
